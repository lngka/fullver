%\Tentwurf\null\Tnewpage
\Tthiweeksheetheader{02.11.17}{4}{4}

\subsubsection{\"Ubungsaufgaben:}
Aufgaben 2.19 ($10+15=25$ Punkte),
2.25 ($15+10=25$ Punkte),
3.1 (20 Punkte)
und
3.7 ($20$ Punkte) aus dem Skript
sowie die Aufgabe~\ref{ex:wz4} ($10$
Punkte) unten.
Abgabe Ihrer L\"osungen:
Sp\"atestens am Donnerstag, den 09.11.17,
um 11 Uhr in den \"Ubungsf\"achern
zur Vorlesung.

\subsubsection{Zeitliche Verlegung des offenen Inforaums:}
Der offene Inforaum findet ab sofort dienstags von
10.45 Uhr bis 12.15 Uhr in
Raum 1056$\,$N statt.

\begin{Twproblem}{(Average-Case-Laufzeit von
Sortieren durch Einf\"ugen)}
\label{ex:wz4}%
In der Vorlesung am 26.10.17 haben 25.10.17 wir die
Average-Case-Laufzeit von
Sortieren durch Einf\"ugen bei einer
Eingabegr\"o\ss e von $n=3$ analysiert
und herausgefunden, dass der Erwartungswert der
Gr\"o\ss e $\sum_{k=2}^n t_k$ bei Eingabe
einer uniform verteilten zuf\"alligen
Permutation von $\{1,2,3\}$ genau
$3/2$ betr\"agt.
Untersuchen Sie, ob die allgemeine Formel
des Skriptes f\"ur die
Average-Case-Laufzeit von
Sortieren durch Einf\"ugen mit diesem
speziellen Ergebnis \"ubereinstimmt.
Begr\"unden Sie Ihre Antwort.
\end{Twproblem}

\newpage
\begin{Tproblemsection}
\renewcommand{\label}[1]{\ignorespaces}
\noindent{\bf Auszug aus dem Skript:}
\end{Tproblemsection}

\newpage
\begin{Tproblemsection}
\renewcommand{\label}[1]{\ignorespaces}
\noindent{\bf Auszug aus dem Skript:}
\end{Tproblemsection}
